\documentstyle[12pt]{article}
\hoffset -0.5truein
\voffset -0.5truein
\textheight 8.0truein
\textwidth  7.0truein
\newcommand{\pr}{{\it precess}~}
\newcommand{\prompt}{Precess$>$:~}
\begin{document}

\title {The \pr program}

\author{Jonathan McDowell}

\date{1989 Feb 27}


\maketitle




\section{Introduction}

\pr is an interactive astronomical coordinate conversion
program. It allows not only precession of equatorial coordinates
but conversion between equatorial, ecliptic, galactic and supergalactic
coordinates.

\section{ Using \pr interactively}

Type \pr to run the program in interactive mode.
You will enter the \pr {\bf command level} which allows you to set the
program parameters. You know you are at the command level when you see the
prompt
\begin{quote}
\prompt
\end{quote}

To see a list of available commands, type {\bf ?} or {\bf help}. The most
usual command sequence is to use {\bf from, to}, and {\bf convert},
 which may be abbreviated
{\bf f},{\bf t}, and {\bf c}. {\bf from} and {\bf to} 
define the input and output coordinate systems,
and {\bf c} enters the {\bf processing level}. 
Example 1: Precess equatorial coordinates from 1950 to 1986. \begin{quote}
\prompt f B1950 t B1986

\prompt c \end{quote}

Example 2: Convert galactic coords to J2000 equatorial coordinates \begin{quote}
\prompt f g t j;c

or alternatively and equivalently,

\prompt from galactic to J2000

\prompt convert \end{quote}


Once you type {\bf c} or {\bf convert}, you will be 
in the processing level, and you will see a message explaining the coordinate
input format and then the processing level prompt. The processing level
prompt is the name of the input system's `x-coordinate'. You then 
type in the coordinates in the appropriate format, either both coordinates
on the same line or x-coordinate on one line and y-coordinate on the next.
In the latter case you will be prompted with the name of the y-coordinate
at the beginning of the second line.


Example 1: entering a pair of equatorial coordinates\begin{quote}
RA (B1950): 02 20 20.1 -00 00 23.3
\end{quote}
Example 2: entering a pair of galactic coordinates\begin{quote}
L: 97.74 -60.18
\end{quote}

RA and Dec are assumed by default to be in the format of Example 1,
i.e. hh mm ss.ss dd mm ss.ss, while other coordinate systems
are assumed by default to be in the format of Example 2, i.e.
in decimal degrees.

The program then writes out the input coordinates and corresponding
output coordinates. You are recommended to check the input coordinates
against the ones you typed in to verify that they were correctly read.

To leave the processing level and return to the command level, type
{\bf q} or {\bf quit} in response to the processing level prompt.
{\bf q} or {\bf quit} issued at the command 
level will leave the program entirely.


There is one other command; {\bf list} or {\bf l} displays the
current settings of the program.

\section{Available coordinate systems}

\subsection{Equatorial coordinates}

Equatorial coordinates consist of a longitudinal Right Ascension (RA,$\alpha$) and 
a latitudinal Declination
(Dec, $\delta$). The plane of zero declination is the projection of 
Earth's equator onto the celestial sphere, and the zero of RA is marked
by the intersection of the ecliptic (Earth orbital) plane with the equatorial
plane. This definition depends on epoch because of precession; its practical 
implementation also depends on the set of fundamental reference stars used
to measure coordinates. There are two fundamental frames of reference
currently in use; the FK4 and FK5 systems. The FK4 system is tied
to the equatorial coordinate system for Besselian epoch B1950.0, while
the more accurate FK5 system currently
being introduced is tied to the equatorial system for Julian epoch J2000.0.
While the coordinate system in most widespread use among astronomers is the B1950.0 system
of equatorial coordinates, the IAU have recommended that J2000.0 coordinates be used
from now on. For example, Space Telescope proposals are required in J2000 coordinates.

Note that since the FK4 and FK5 reference frames rotate relative to one another,
the transformation from B1950 to J2000 affects not only the coordinates but
the proper motions of stars. An object with zero proper motion in B1950 will
have a nonzero proper motion in J2000 and vice versa. Since this program does not deal with 
proper motions, coordinate conversions have been adopted which assume that
the object in question has zero proper motion in the J2000 system.  (Since
it was designed with extragalactic objects in mind which have effectively
zero proper motion in the real world, and J2000 is meant to be a better approx
to the real world than B1950, this is the sensible thing to do.) If you want to
deal with proper motions and need to do proper positions (eg for pulsar
astrometry or something), there are FORTRAN routines which can do this, available
from this author (mail to mcdowell@cfa). 

Note further, however, that the difference between J2000 and B2000 is typically
less than one arc second. High Energy astrophysicists, therefore, need not worry
about the difference. But we might as well use the Right Thing, which IAU deems
to be J2000.

The default FROM and TO systems for \pr are B1950 and J2000 respectively.
Precession from an arbitrary B epoch to and arbitrary J epoch is done 
in three stages, Bxxxx to B1950 to J2000 to Jyyyy, but this is transparent to
the user. Precession from Bxxxx to Byyyy is done directly.

To select a coordinate system, give the name of the epoch, e.g.
{\bf FROM B1979.0} or {\bf FROM J1990.0}. The format of the 
numerical epoch is unimportant (i.e. B1979=B1979.0=b1979.00) but there
must not be any spaces between the B or J and the number. 

B and J are provided
as synonyms for B1950 and J2000. C is provided as a further synonym for B1950
for people who think of this system as 'celestial' coordinates.

\subsection{Galactic coordinates}

Defined conceptually by the Galactic plane and the Galactic centre,
galactic longitude $l$ and galactic latitude $b$ are the IAU 1958 system, formerly
called $l_{II}$ and $b_{II}$. Noone should be using the old ones by now so I have ignored
them. The system is defined in terms of B1950 equatorial coordinates as
RA of (l=0) = 192.25 degrees, inclination of galactic equator to B1950 equator = 62.6 
degrees, longitude of ascending node 33 degrees. 

\begin{center}
\begin{tabular}{rrrrrr}
\multicolumn{2}{c}{Galactic l,b}
&\multicolumn{2}{c}{ B1950 RA,Dec }
&\multicolumn{2}{c}{ J2000 RA,Dec }\\
\hline
0&0& 17 42 26.58 &-28 55 00.43 &17 45 37.20&-28 56 10.22\\
0&+90& 12 49 00.00 &+27 24 00.00 &12 51 26.28 &+27 07 41.70\\
33.0&0&18 49 00.00&00 00 00.00&18 51 33.73& +00 03 38.13\\
123.0&+27.4&00 00.00 00&+90 00 00.00&12 01 16.85&+89 43 17.74\\
\end{tabular}
\end{center}

The J2000 north celestial pole is at l=122.932, b=27.128.

To select galactic coordinates, use from/to option {\bf G} or {\bf Galactic}.

\subsection{Supergalactic coordinates}

With the advent of large scale structure studies, the supergalactic
coordinate system introduced by de Vaucouleurs is coming into more
widespread use. The supergalactic equator is conceptually defined
by the plane of the local (Virgo-Hydra-Centaurus) supercluster, and the
origin of supergalactic longitude is at the intersection of the 
supergalactic and galactic planes.
Supergalactic longitude and latitude SGL,SGB are
defined by
\begin{quote}
SGL=0 at l=137.37

SGB=90 at l=47.37 b=6.32
\end{quote}
so
\begin{center}
\begin{tabular}{rrrrrr}
\multicolumn{2}{c}{  SGL,SGB}
&\multicolumn{2}{c}{Galactic l,b}
&\multicolumn{2}{c}{ J2000 RA,Dec }\\
 \hline
0&0& 137.37& 0 & 02 49 14.43&+59 31 42.05\\
0&+90& 47.37& +6.32& 18 55 00.98 & +15 42 32.17\\
90.0&6.32 & 0& +90&12 51 26.28&+27 07 41.70\\
\end{tabular}
\end{center}

To select supergalactic coordinates, use from/to option {\bf SG}.


\subsection{Ecliptic coordinates}

Ecliptic coordinates are defined by the earth's orbital plane.
The B1950 north ecliptic pole is at B1950 RA 18 00 00, Dec +66 33 15
The zeroes of ecliptic longitude and B1950 RA coincide.

To select ecliptic coordinates, use from/to option {\bf EC}.

Ecliptic coordinates are weakly epoch dependent, so you may wish
to use a specific epoch to define the orbital plane rather than 
the default B1950.0. To select Besselian epoch Bxxxx, use
the from/to option ECxxxx. Julian epochs are not provided.
EC is equivalent to EC1950.

\subsection{Constellations}

You can get the constellation in which a position lies.
For example: 
precess f b t con
will prompt you for B1950 positions. They are precessed
to B1875.0 and compared with the Delporte (1935) constellation
definitions which are read in from a library file, \$JCMLIBDATA/Constellations
and \$JCMLIBDATA/Constellation.parallels.

\subsection{Changing the output format}

Coordinates can be read an written either in 
sexagesimal (``Shh mm ss.ss'') or decimal degree (``Sddd.ddd'')
formats. Each coordinate system adopts one format as default
(sexagesimal for equatorial coordinates, decimal degree for
galactic coordinates, etc.) To override the default, specify
`/DEG' or `/HMS' appended to the coordinate system, for instance 
{\bf FROM B1979.0/DEG} or {\bf TO G/HMS}.

\section{Using \pr on a file}

You can convert a file of coordinates if it is in a format
that \pr can read. Here is an example file format when
the input system uses ``hh mm ss.ss'' format:

\begin{verbatim}
00 01 12.3 -00 02 23.4
02 59 59.9  23 12 22.1
02 23 0  11 23 12
23 48 48.23 -2 11 14.123
\end{verbatim}

To transform this file from B1950 to galactic coordinates,
type \begin{quote} \pr {\it t g:bfile:gfile}\end{quote}
where {\it bfile} is the file and {\it gfile} is the output file,
which will contain:

\begin{verbatim}
RA  (B1950)         Dec (B1950)         L                   B
00 01 12.30   -00 02 23.40               98.275        -60.331
02 59 59.90    23 12 22.10              158.172        -30.327
02 23 00.00    11 23 12.00              156.571        -45.016
23 48 48.23   -02 11 14.12               90.687        -60.963
23 48 48.23   -02 11 14.12               90.687        -60.963
\end{verbatim}

The general syntax is 
\begin{quote} \pr  commands:infile:outfile \end{quote}
where {\it commands} is a string of \pr commands separated
by spaces, and the input and output files default to 
standard input and output. The {\bf c} command is assumed automatically,
as are {\bf q} commands at the end, so {\it commands} usually
consists of a {\bf from/to} specification. The default is B1950 to J2000,
so to convert the above file to J2000 would require
\begin{quote} \pr {\it :bfile:jfile}\end{quote}
Note the colons separating the absent first
argument from the second (input file) argument.
To go from B1979.0 to B1950.0, you might use
\begin{quote} \pr {\it f B1979 t B1950:bfile:newfile}\end{quote}


This syntax is also useful for interactive mode, to go directly into
the processing level. For example:
\begin{quote} \pr {\it from g to b list}\end{quote}
where the {\bf list} command makes the program echo
the choice of input and output system to the terminal prior to
the start of processing.

\section{Bugs}

Please report all bugs to me, preferably by email (jmcdowell@cfa.harvard.edu).
Comments on the documentation are also welcome.
\end{document}

\documentclass[12pt]{article}
\hoffset -0.5truein
\voffset -0.5truein
\textheight 8.0truein
\textwidth  7.0truein
\newcommand{\pr}{{\it colden}~}
\newcommand{\prompt}{Colden[Setup]$>$:~}
\begin{document}

\title {The \pr program}

\author{Jonathan McDowell}

\date{1997 Aug 4}


\maketitle




\section{Introduction}

\pr is an interactive program to evaluate the neutral hydrogen
column density at a given direction on the sky.

\section{ Using \pr interactively}

Type \pr to run the program in interactive mode.
You will enter the \pr {\bf command level} which allows you to set the
program parameters. You know you are at the command level when you see the
prompt
\begin{quote}
\prompt
\end{quote}

To see a list of available commands, type {\bf ?} or {\bf help}. The most
usual command sequence is to use {\bf data } to select the
dataset and {\bf convert } to enter the main calculation loop. These
 may be abbreviated
{\bf d} and {\bf c}.  You may also use {\bf f} or {\bf from} to
select a new input coordinate system (B1950 is the default; see the
{\bf precess} documentation for choices).
Once you type {\bf c} or {\bf convert}, you will be 
in the processing level, and you will see a message explaining the coordinate
input format and then the processing level prompt. The processing level
prompt is the name of the input system's `x-coordinate'. You then 
type in the coordinates in the appropriate format, either both coordinates
on the same line or x-coordinate on one line and y-coordinate on the next.
In the latter case you will be prompted with the name of the y-coordinate
at the beginning of the second line.


Example 1: entering a pair of equatorial coordinates\begin{quote}
RA (B1950): 02 20 20.1 -00 00 23.3
\end{quote}
Example 2: entering a pair of galactic coordinates\begin{quote}
L: 97.74 -60.18
\end{quote}

RA and Dec are assumed by default to be in the format of Example 1,
i.e. hh mm ss.ss dd mm ss.ss, while other coordinate systems
are assumed by default to be in the format of Example 2, i.e.
in decimal degrees.

The program then converts the input coords to B1950 (if needed) and
galactic, and calculates the estimated column density. These results
are printed, and you are then prompted for more coordinates.

To leave the processing level and return to the command level, type
{\bf q} or {\bf quit} in response to the processing level prompt.
{\bf q} or {\bf quit} issued at the command 
level will leave the program entirely.

There are two datasets supported:
\begin{itemize} 
\item {\bf Bell} is the Stark et al Bell Labs survey. We have an
early version of the spectra from 1984, and have not checked them
in detail against the published (1992) FITS data; we hope to support the
published version in the future. There are two data files, one integrated
over velocity and one with the velocity-resolved spectra.
To select a velocity slice in the Bell data, use the VLIMS vmin vmax
command at the command level. VLIMS * returns the velocity slice
to its default, maximum range, -550 to +550 km/s. The program will not use
the velocity-resolved data file if this default is in effect, and
in fact you don't have to have the file there at all (useful since it's
quite big). 

{\bf Note that the Stark et al data only covers $\delta>-40$ degrees}.

\item {\bf NRAO} is the Dickey and Lockman (1990) all-sky interpolation
of Stark et al and several other surveys. It is not velocity-resolved.

\end{itemize}

The P command sets the output format: you can type P0 (terse), P1 (normal)
or P2 (verbose).
There is one other command; {\bf list} or {\bf l} displays the
current settings of the program. 


\section{Using \pr on a file}

You can convert a file of coordinates if it is in a format
that \pr can read. Here is an example file format when
the input system uses ``hh mm ss.ss'' format:

\begin{verbatim}
00 01 12.3 -00 02 23.4
02 59 59.9  23 12 22.1
02 23 0  11 23 12
23 48 48.23 -2 11 14.123
\end{verbatim}

To calculate NH for this file if the coords are B1950, using the Bell data
(default)
type \begin{quote} \pr {\it :bfile:gfile}\end{quote}
where {\it bfile} is the file and {\it gfile} is the output file.

The general syntax is 
\begin{quote} \pr  commands:infile:outfile \end{quote}
where {\it commands} is a string of \pr commands separated
by spaces, and the input and output files default to 
standard input and output. The {\bf c} command is assumed automatically,
as are {\bf q} commands at the end, so {\it commands} usually
consists of a {\bf data} or {\bf vlims} specification. 
The default is B1950 coords and the Bell dataset,
so if the above file is J2000 and you want to use the NRAO dataset,
\begin{quote} \pr {\it from J2000 data NRAO:bfile:jfile}\end{quote}
Note the colons separating the absent first
argument from the second (input file) argument.
To select a velocity slice in the Bell dataset, you might use
\begin{quote} \pr {\it vlims -20 120:bfile:newfile}\end{quote}


This syntax is also useful for interactive mode, to go directly into
the processing level. For example:
\begin{quote} \pr {\it from g data nrao list}\end{quote}
where the {\bf list} command makes the program echo
the choice of input and output system to the terminal prior to
the start of processing.


\section{Using \pr for a single evaluation}

The EVAL command evaluates a single position and exits.
Example:
\begin{verbatim}
colden data nrao eval 14 11 30 20 11 10
\end{verbatim}

\section{Bugs}

Please report all bugs to me, preferably by email (jcm@urania.harvard.edu)
Comments on the documentation are also welcome.
\end{document}

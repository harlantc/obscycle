\documentclass{article}
\usepackage{longtable}
\usepackage{lscape}
\setlongtables
\newenvironment{temp}{\begin{quote}\bf}{\end{quote}}
\newenvironment{rtlist}[1]{
\small
\begin{longtable}{p{5.1in}p{0.8in}}
\hline
\multicolumn{2}{c}{#1}\\
\hline
\endhead
\hline
\endfoot
}{\end{longtable}}
\newenvironment{crtlist}[1]{
\small
\begin{longtable}{p{3.1in}p{2.8in}}
\hline
\multicolumn{2}{c}{#1}\\
Fortran & C\\
\hline
\endhead
\hline
\endfoot
}{\end{longtable}}

\textheight 9.0in
\textwidth 7.0in
\hoffset -1.0in
\voffset 0.0in

\begin{document}

\large


\title{The DATES program: version 2.0}

\author{Jonathan McDowell}

\date{2008 Jun 16}

\maketitle

\newpage

\normalsize


\tableofcontents
\section{Introduction}

The DATES program allows you to convert between different 
calendars and timescales. DATES is one of the JCMUTILS
family of utility programs, all of which share similar
interfaces. The default use of DATES is to convert from
a standard Gregorian calendar date to the Julian Day Number.
For example, a Gregorian date of 1997 Apr 27 14:03 UTC
corresponds to JD 2450566.0854167 UTC.

To run dates on the HEAD network, use
\begin{verbatim}
setenv JCMPATH /proj/jcm/src/libdata
alias dates /proj/jcm/src/bin/dates
\end{verbatim}

DATES has four main command line modes: interactive setup,
interactive convert,
single-evaluation convert, and batch.

The syntax for DATES is:
\begin{quote}
dates {\it commands : infile : outfile}
\end{quote}

The colon character is special for JCMUTILS programs, and
must be escaped if it appears within the {\it commands} text.
If {\it infile} is present, DATES enters batch mode and
reads a series of dates from infile, converting each one
and printing the output in outfile (which defaults to the
terminal). If {\it commands} are missing, the default
conversion of Gregorian date UTC to JD(UTC) is used. Example:
\begin{quote}
dates from EST to TAI:date\_list.in: date\_list.out
\end{quote}
where date\_list.in contains, for example:
\begin{verbatim}
1993 Jun 30 06:00
1993 Jun 30 23:30
1993 Jul  1 00:30
1993 Jun 30 18:59:59
1993 Jun 30 18:59:60
1993 Jun 30 19:00:00
1993 Jun 30 23:59:59
1993 Jul  1 00:00:00
\end{verbatim}
Then date\_list.out will contain the same dates converted
from Eastern Standard Time (EST) to International Atomic Time (TAI):
\begin{verbatim}
EST                                              TAI
Wed AD 1993 Jun 30  06:00       EST (Gregorian)  Wed AD 1993 Jun 30  11:00:27.00 TAI (Gregorian)
Wed AD 1993 Jun 30  23:30       EST (Gregorian)  Thu AD 1993 Jul  1  04:30:28.00 TAI (Gregorian)
Thu AD 1993 Jul  1  00:30       EST (Gregorian)  Thu AD 1993 Jul  1  05:30:28.00 TAI (Gregorian)
Wed AD 1993 Jun 30  18:59:59    EST (Gregorian)  Thu AD 1993 Jul  1  00:00:26.00 TAI (Gregorian)
Wed AD 1993 Jun 30  18:59:60    EST (Gregorian)  Thu AD 1993 Jul  1  00:00:27.00 TAI (Gregorian)
Wed AD 1993 Jun 30  19:00:00    EST (Gregorian)  Thu AD 1993 Jul  1  00:00:28.00 TAI (Gregorian)
Wed AD 1993 Jun 30  23:59:59    EST (Gregorian)  Thu AD 1993 Jul  1  05:00:27.00 TAI (Gregorian)
Thu AD 1993 Jul  1  00:00:00    EST (Gregorian)  Thu AD 1993 Jul  1  05:00:28.00 TAI (Gregorian)
\end{verbatim}


The special command EVAL forces single-evaluation mode:

\begin{verbatim}
dates eval 1993 Jul 1 00\\:30 
dates eval 1993 Jul 1 0030 
\end{verbatim}
will both convert the given date to a Julian Day. The first form
escapes the colon in the time using a double backslash (one for Unix,
another for JCMUTILS).

With no infile or outfile, we enter interactive convert mode:

\begin{verbatim}
dates from Greg to JD
\end{verbatim}
gives a prompt to which one may enter dates one at a time and
see the conversion. To leave the program, one types 'q' twice.
Why twice? Because the first time  you leave interactive
convert mode to interactive setup mode, which allows you
to change the conversion parameters.
If you just enter DATES with no arguments:

\begin{verbatim}
dates
\end{verbatim}
you enter interactive setup mode first. You get the prompt
\begin{verbatim}
Dates [Setup]>:
\end{verbatim}
Enter ? to get a brief summary of setup commands. The most 
important one is C, the convert command which pushes you
into the interactive convert loop.

\section{Setup commands}

The following commands can be used on the command line
or in setup mode.

\begin{longtable}{llll}
Command & Short form & Syntax & Action\\
\hline
HELP    & H, ?      & H       & Brief summary of commands\\
LIST    & L         & L/opt   & Display internal program state\\
PRINT   & P         & Pn      & Set output format type \\
FROM    & F         & F spec  & Define source calendar and timescale\\
TO      & T         & T spec  & Define destination calendar and timescale\\
EVAL    & E         & E date  & Convert once and exit\\
CONVERT & C         & C       & Enter interactive convert loop\\
DO      & D         & DO date1;date2;step& Loop convert\\
QUIT    & Q         & Q       & Exit program\\
\end{longtable}

\subsection{Display commands}

\begin{itemize}

\item HELP prints a brief command summary like the one above.

\item LIST has several options. With no option, LIST
displays the current conversion settings (from and to). 

\item LIST/CONV displays the current conversion settings
in greater detail.

\item LIST/CAL lists the available calendars, which may be
selected with the FROM and TO commands.

\item LIST/TS lists the available timescales, which also may
be selected with the FROM and TO commands.

\item LIST/CT and LIST/TT list the internally available calendar
types and timescale types, useful mainly for debugging.

\item PRINT n or Pn sets the print mode to n, where n is
one of the integers 0, 1, 2.
\begin{itemize}
\item P2 is the default for interactive mode. 
Each conversion outputs four lines:
a leading banner of dashes, the 'from' info (to check the
program has parsed your input correctly), the 'to' info
(the result), and a trailing banner of dashes. Each 'info'
gives the calendar followed by the value.

\item P1 is the default for batch mode. A single line
gives the from and to values side by side, and one extra
line is issued on entry to convert mode (i.e. a header line)
giving the from and to calendars.

\item P0 is a concise mode which gives only the output value.
\end{itemize}

\end{itemize}

\subsection{Specification commands}

The FROM and TO commands set up the conversion. For each end
of the conversion you can specify a calendar and a timescale,
although some calendars do not support timescales. Essentially,
the calendar defines how each date is represented, and the
timescale defines the precise interpretation of the date
as an absolute time relative to, say, TAI.

The default input calendar is the Gregorian calendar, which
is the usual civil calendar in the US and Europe. The default
output calendar is the Julian Day number. The default input and output
timescales are both UTC (Coordinated Universal Time), which
is the usual scientific timescale. Details of calendars and
timescales are given in a later section.

A spec for FROM or TO consists of:
\begin{itemize}
\item The name of a calendar, or
\item The name of a timescale, or
\item The name of a calendar followed by the timescale name in
parentheses.
\end{itemize}

If only a timescale is given with no calendar, the Gregorian
calendar is assumed. If a calendar but no timescale is given, the current
timescale is retained.

The command 
\begin{verbatim}
FROM GREG(UTC) TO JD(UTC)
\end{verbatim}
restores the default setting.
The command
\begin{verbatim}
TO TAI
\end{verbatim}
would change the output system to GREG(TAI).
If this was followed by the command
\begin{verbatim}
TO MJD
\end{verbatim}
the output system would change to MJD(TAI).


Calendars supported are:
\begin{longtable}{lp{3.0in}ll}
Specifier & Calendar   & Type  & Output only?\\
\hline
GREG      & Gregorian date  & Gregorian calendar\\
DATE      & Gregorian date with simplified output & Gregorian calendar\\
PACK      & Gregorian date in packed format &  Gregorian calendar\\
DOY       & Gregorian date in packed format, day of year& Gregorian calendar\\
\\
JD       & Julian Day Number  & JD \\
MJD      & Modified Julian Day & JD\\
DAYS()   & Days since ZERO     & JD\\
TIME()   & Seconds since ZERO  & Elapsed time\\
\\
GSD       & Greenwich Sidereal Date & GSD & Yes\\
GST       & Greenwich Sidereal Time & GSD & Yes\\
\\
OS        & Old Style Julian Calendar & Archaic: Julian proleptic\\
ROMAN     & Roman calendar (post-Augustan)& Archaic: Roman & Yes\\
RF        & French Revolutionary calendar& Archaic: French & Yes\\
\hline
\end{longtable}
Note that some calendars may only be used as the output system.

Timescales supported by default are:
\begin{longtable}{lp{3.0in}}
Specifier & Timescale   \\
\hline
UTC       & Coordinated Universal Time \\
TT        & Terrestrial Time (Ephemeris)\\
TDB       & Barycentric Dynamical Time \\
TAI       & Coordinated Atomic Time    \\
UT1       & Universal Time (output only) \\
GMST      & Sidereal Time (output only) \\
\hline
\end{longtable}
as well as local civil time defined as an offset from
UTC. In addition, if the JCMLIBDATA library data is
available (the environment variable JCMPATH must be
set, or the library data must be at the location
in the source tree which is 
initialized by compiling the JCMUTILS library), the following
additional timescales are defined as offsets to UTC:

\begin{verbatim}
MST         Moscow Summer Time      +0400
DMV         Moscow Decree Time      +0300
BST         British Summer Time     +0100
GMT         Greenwich Mean Time     +0000
EDT         Eastern Daylight Time   -0400
EST         Eastern Standard Time   -0500
CDT         Central Daylight Time   -0500
CST         Central Standard Time   -0600
MDT         Mountain Daylight Time  -0600
MST         Mountain Standard Time  -0700
PDT         Pacific Daylight Time   -0700
PST         Pacific Standard Time   -0800
\end{verbatim}

It would be nice if the software knew when to convert historically
from summer time to standard time, but the rules have changed over
the years and from place to place.

The special calendars DAYS and TIME
give elapsed time since a zero point. The default
zero point is the origin of MJD, but you can redefine the
zero point, creating an
alternate 'pseudo-MJD' by
\begin{verbatim}
Dates [Setup]>: FROM GREG TO DAYS(JD 2440000.5); CONVERT
Gregorian date:1994 Aug 13
--------------------------------------------------------------------------------
Gregorian date (UTC)     Sat AD 1994 Aug 13              UTC (Gregorian)
Days(since JD 2440000.5 UTC, 1968 May 24 )   9577.0000000000
--------------------------------------------------------------------------------
\end{verbatim}
This particular zero point is frequently used among the X-ray binary
community.

Alternatively, one might define a zero point using a Gregorian date:
\begin{verbatim}
Dates [Setup]>: FROM GREG TO TIME(1994 Jan 0.0); CONVERT
Gregorian date (UTC): 1994 Aug 13 14:20:22.3
--------------------------------------------------------------------------------
Gregorian date (UTC)     Sat AD 1994 Aug 13  14:20:22.30 UTC (Gregorian)
Time(since JD 2449352.5 UTC, 1993 Dec 31 ) 19491623.3
--------------------------------------------------------------------------------
\end{verbatim}
which gives the true elapsed number of seconds since the given zero
point. (Note that leap seconds are correctly taken into account).


\subsection{Conversion commands}

The EVAL command prompts the user for a date in the input
system, does the conversion, and then exits the program.

The DO command does a 'do loop', listing regularly spaced
dates between a start and an end date. The step can be expressed
in YR (years), D (days), H (hours), M (minutes) or S (seconds).
Example:
\begin{verbatim}
FROM UTC TO TAI; P1
DO 1993 Jun 30 23:59:59;1993 Jul 1 00:00:02; 0.2 s
DO 1992 Apr 1; 1998 Apr 1; 2 yr
\end{verbatim}
This is a little tricky as the `regular' increments are
not really regular - leap years, leap seconds, etc. make
things interesting to calculate, but I think most of the
bugs are out.

The CONVERT command enters the interactive convert loop. The user
is prompted for a date, the conversion is output, and then
the user is prompted for another date.

If the input calendar is GREG, in the convert loop the
user may omit the date on second and subsequent calls,
just giving the time of day:
\begin{verbatim}
Gregorian date:1994 Aug 13 14:20
--------------------------------------------------------------------------------
Gregorian date          Sat AD 1994 Aug 13  14:20       UTC (Gregorian)
Julian Day              JD 2449578.0972222222 UTC
--------------------------------------------------------------------------------
Gregorian date:14:22
--------------------------------------------------------------------------------
Gregorian date          Sat AD 1994 Aug 13  14:22       UTC (Gregorian)
Julian Day              JD 2449578.0986111111 UTC
--------------------------------------------------------------------------------
\end{verbatim}

In this case, the colon-separated format for the time is mandatory,
so the code can tell that a time of day and not a year is not being entered.

\section{Input formats}

\subsection{JD}

For the JD, MJD and DAYS calendars, one must simply enter a
numeric value. However, the internals of JCMCAL split
off the integer part into a 4 byte integer, so only
Julian days between about $-2\times10^9$ and $2\times10^9$
(i.e. 5 million years BC to 5 million years AD) are supported.

\subsection{GREG}

The Gregorian calendar supports a variety of input formats.
The DATE calendar is identical to Gregorian but suppresses
the calendar name and the day of the week in its output.

The standard form of the input for GREG is:
\begin{verbatim}
AD 1993 Jun 30 23:59:60.234 
\end{verbatim}
The words AD or BC, followed by the year, the month name, the
day, and the time of day with hours, minutes and seconds
separated by colons.

In addition:
\begin{itemize}
\item For BC dates, the BC is mandatory; but for AD dates,
the AD may be omitted.

\item We use the usual astronomical convention in which
day 0 of a month is the last day of the preceding month.
Thus May 0 is Apr 30.

\item Month numbers may be used instead of month names:
e.g. 1993 06 30 23 59 60.234.

\item Alternate formats for the time are supported:
colons may be omitted, spaces may be used instead, or the
letters h,m,s:
\begin{verbatim}
23:59:60.234
235960.234
23 59 60.234
23h59m60s.234
\end{verbatim}


\item For imprecise dates, trailing information may be omitted, as for example:
\begin{verbatim}
1993 Jun 30 23:59:60
1993 Jun 30 23:59  or 1993 Jun 30 2359
1993 Jun 30 23h
1993 Jun 30 
1993 Jun
1993
\end{verbatim}
In these cases, the earliest possible date is assumed, so
1993 Jun means 1993 Jun 0 (i.e. 1993 May 31.0) and 1993 means
1993 Jan 0 (i.e. 1992 Dec 31.0). The intent is that
if a GREG to JD conversion is used, and the resulting JDs are
sorted, they will come out with the imprecise dates first.

\item Further forms of imprecise and uncertain dates are
supported:
\begin{longtable}{ll}
1993 Jun 30 23:59?   & (Centiday)\\
1993 Jun 30?         & (Approx day)\\
1993 Jun?            & (Approx month)\\
1993 Q2              & (2nd quarter of 1993)\\
1993?                & (Approx year)\\
c. 1993, ca. 1993    & (Approx year)\\
1990s                & (Decade)\\
1990s?               & (Approx decade)\\
20C, BC 17C          & (Century)\\
20C?, ca. BC 17C     & (Approx century)\\
BC 3M                & (Millenium)\\
\end{longtable}

\item Day of Year format supported on input:
\begin{verbatim}
1997 Day 104 14:20:32
\end{verbatim}


\end{itemize}

\subsection{DOY}

The DOY (Day of year) format is a compact numeric format
representing the year and day of year. The last three digits
of the integral part of the number are taken to be the day
of year; earlier digits are then assumed to be the year.
If the year is less than 100, 1900 is automatically added.
Examples:
\begin{longtable}{ll}
DOY &  GREG \\
97212.324     & 1997 Jul 31 07:46:33.6\\
1997212.324   & 1997 Jul 31 07:46:33.6\\
2002004       & 2002 Jan  4 00:00\\
02004         & 1902 Jan  4 00:00\\
102004        &  102 Jan  4 00:00\\
\end{longtable}

The DOY calendar is the only way to get DOY values on output,
but on input one may also use the GREG calendar in the form
1997 Day 212 07:46.


\subsection{PACK}

Similar to the DOY calendar, the PACK calendar retains
the idea of month and day.

Examples:
\begin{longtable}{ll}
PACK &  GREG \\
970731.324     & 1997 Jul 31 07:46:33.6\\
19970731.324   & 1997 Jul 31 07:46:33.6\\
20020104       & 2002 Jan  4 00:00\\
020104         & 1902 Jan  4 00:00\\
1020104        &  102 Jan  4 00:00\\
\end{longtable}


\subsection{TIME}

The elapsed time calendar TIME gives seconds since
the zero point.
A numeric value must be entered.

\subsection{GSD, GST}

The Greenwich Sidereal Date is essentially JD(GMST),
i.e. the Julian Day number for Greenwich Mean Sidereal
Time. Calendar GSD outputs the result as a JD and
fraction of day, while GST outputs the result as
an integer day number and hh:mm:ss.sss of sidereal
time.

\subsection{OS}

The Old Style (OS) calendar, or Julian proleptic calendar, 
was the civil calendar in Italy until the 16th century,
in England until the 18th century, and in Russia until
the 1920s. It differs from the Gregorian (New Style) calendar in that
century years are always leap years, i.e. every year
divisible by 4 is a leap year.

The Old Style and New Style calendars were in agreement
from AD 200 Mar 1 (JD 1794167.5) to AD 300 Feb 28 (JD 1830690.5). 
Input format for OS is the same as for GREG.

\subsection{ROMAN}

The Roman imperial proleptic calendar implemented here
is based on the Julian proleptic calendar. The Julian calendar
date is converted to time since the founding of the city of
Rome ('ab urbe condita', AUC), and the days and months
are converted to Roman imperial (post-Augustan) days and months. 
Leap days are handled by the insertion of a bissextile day,
BIS-KAL. FEB, on Feb 23, between VI KAL MAR and V KAL MAR.
The eponymous consuls are listed if the appropriate data
file is in the path.

\subsection{RF}

The calendar of the first Republique Francaise is supported.
There may be some bugs.


\section{XRCF Examples}

Here are some examples for the AXAF calibration data:

IRIG time to calendar date:
\begin{verbatim}
% dates to greg p0 eval 1996 Day 356 
Sat AD 1996 Dec 21  00:00:00.00 UTC (Gregorian)
\end{verbatim}

IRIG time to local date:
\begin{verbatim}
% dates to greg/CST eval 1996 Day 356 042232.21
--------------------------------------------------------------------------------
UTC                     Sat AD 1996 Day 356 04:22:32.21 UTC (Gregorian)
CST                     Fri AD 1996 Dec 20  22:22:32.20 CST (Gregorian)
--------------------------------------------------------------------------------
\end{verbatim}

IRIG time to seconds from 1994.0
\begin{verbatim}
% dates
Dates [Setup]>:to time(1994.0)
Dates [Setup]>:eval 1996 Day 356 04:22:32.212
--------------------------------------------------------------------------------
Gregorian date          Sat AD 1996 Day 356 04:22:32.21 UTC (Gregorian)
Time(since JD 2449352.5 UTC, 1993 Dec 31 ) 93846154.212
--------------------------------------------------------------------------------
\end{verbatim}

Seconds from 1994.0 to calendar date
\begin{verbatim}
% dates
Dates [Setup]>:from time(1994.0) to greg
Dates [Setup]>:eval 93846154.212
--------------------------------------------------------------------------------
TIME(JD 2449352.5)      93846154.212
Gregorian date          Sat AD 1996 Dec 21  04:22:32.21 UTC (Gregorian)
----------------------------------------------------------------------------------------------------------------------------------------------------------------
\end{verbatim}

Seconds from 1994.0 to IRIG
\begin{verbatim}

% dates
Dates [Setup]>:from time(1994.0) to doy
Dates [Setup]>:eval 93846154.212
--------------------------------------------------------------------------------
TIME(JD 2449352.5)      93846154.212
Day of year              1996356.18231727
-------------------------------------------------------------------------------
\end{verbatim}

Calendar date to IRIG


\begin{verbatim}
% dates to doy p0 eval 1996 Dec 21
  1996356.00000000
\end{verbatim}


\end{document}


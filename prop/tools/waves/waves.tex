\documentclass[12pt]{article}
\hoffset -0.5truein
\voffset -0.5truein
\textheight 8.0truein
\textwidth  7.0truein
\newcommand{\pr}{{\it waves}~}
\newcommand{\prompt}{Waves[Setup]$>$:~}
\begin{document}

\title {The \pr program}

\author{Jonathan McDowell}

\date{1997 Aug 4}


\maketitle




\section{Introduction}

\pr is an interactive program to evaluate the neutral hydrogen
column density at a given direction on the sky.

\section{ Using \pr interactively}

Type \pr to run the program in interactive mode.
You will enter the \pr {\bf command level} which allows you to set the
program parameters. You know you are at the command level when you see the
prompt
\begin{quote}
\prompt
\end{quote}

To see a list of available commands, type {\bf ?} or {\bf help}. The most
usual command sequence is to use {\bf from } and {\bf to } to select the
input and output systems and {\bf convert } to enter the main calculation loop. These
 may be abbreviated
{\bf f, t} and {\bf c}.  
Once you type {\bf c} or {\bf convert}, you will be 
in the processing level, and you will see the processing level prompt. The processing level
prompt is the name of the input system. You then
type in the value and the program prints the converted value.
Repeat this until you want to change the setup, type 'q' to get back to the
setup level and 'q' again to leave the program entirely.

The P command sets the output format: you can type P0 (terse), P1 (normal)
or P2 (verbose).

\section{Supported conversions}

The following are valid choices for FROM and TO:
\begin{itemize}
\item A  - wavelength in Angstrom.
\item mu - wavelength in microns
\item nm - wavelength in nanometer
\item keV - energy in keV
\item K - energy in Kelvin  (with h = c = 1 )
\item lognu - log frequency in Hz
\item Hz - frequency in Hz
\end{itemize}


The default input system is lognu. This system also supports
a generic string input, recognizing '100.3 A', '40 mu', '6.4 keV'.
So you often won't bother to do a FROM, just use the
default and select the output system using TO.

\section{Using \pr on a file}

You can convert an ASCII file of values (one value per line)

\begin{verbatim}
waves from A to keV : infile : outfile
\end{verbatim}

The general syntax is 
\begin{quote} \pr  commands:infile:outfile \end{quote}
where {\it commands} is a string of \pr commands separated
by spaces, and the input and output files default to 
standard input and output. The {\bf c} command is assumed automatically,
as are {\bf q} commands at the end, so {\it commands} usually
consists of a {\bf from} and {\bf to} specification. 

This syntax is also useful for interactive mode, to go directly into
the processing level. For example:
\begin{quote} \pr {\it from A to keV }\end{quote}
%where the {\bf list} command makes the program echo
%the choice of input and output system to the terminal prior to
%the start of processing.


\section{Using \pr for a single evaluation}

The EVAL command evaluates a single position and exits.
Example:
\begin{verbatim}
waves from keV to lognu p0 eval 6.4
 or
waves to lognu p0 eval 6.4 keV
\end{verbatim}
Note the use of the p0 command to force terse output.

\section{Bugs}

Please report all bugs to me, preferably by email (jcm@urania.harvard.edu)
Comments on the documentation are also welcome.
\end{document}
